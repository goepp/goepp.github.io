%    File: francis_bacon.tex
%   Created: mar. oct. 15 07:00  2019 C
% Last Change: mar. oct. 15 07:00  2019 C
%
\documentclass[a4paper, 12pt]{report}
\usepackage{ebgaramond}
\usepackage[francais]{babel}
\usepackage[utf8]{inputenc}
\usepackage[T1]{fontenc}
\begin{document}
\topskip0pt
\vspace*{\fill}
\pagestyle{empty}
\begin{center}
		\Huge{Francis Bacon}
\end{center}
\vspace*{\fill}
Je ne dessine pas.
Je commence à faire toutes sortes de taches.
J’attends ce que j'appelle « l'accident » : la tache à partir de laquelle va partir le tableau.
La tache c’est l’accident.
Mais si on tient à l’accident, si on croit qu’on comprend l’accident, on va faire encore de l’illustration, car la tache ressemble toujours à quelque chose.
On ne peut pas comprendre l’accident.
Si on pouvait le comprendre, on comprendrait aussi la façon avec laquelle on va agir.
Or cette façon avec laquelle on va agir, c’est l’imprévu, on ne peut jamais la comprendre : « It’s basically the technical imagination » : « l’imagination technique ».
J’ai beaucoup cherché comment appeler cette façon imprévisible avec laquelle on va agir.
Je n’ai jamais trouvé que ces mots-là : imagination technique.
Vous comprenez, le sujet est toujours le même.
C’est le changement de l’imagination technique qui peut faire se « retourner » le sujet sur le système nerveux personnel.
Imaginez des scènes extraordinaires, ce n’est pas intéressant du tout du point de vue de la peinture, ça n’est pas l’imagination.
L’imagination véritable est construite par l’imagination technique.
Le reste c’est l’imagination imaginaire, ça ne mène nulle part.
Je ne peux pas lire Sade pour cette raison.
Ça ne me dégoûte pas complètement, mais ça m’ennuie.
De même il y a des écrivains mondialement connus que je ne peux pas lire non plus.
Ils écrivent des choses qui sont des histoires sensationnelles, seulement ça.
But they have not the technical sensation.
C’est toujours par les techniciens qu’on trouve les vraies ouvertures.
L’imagination technique c’est l’instinct qui travaille hors des lois pour retourner le sujet sur le système nerveux avec la force de la nature.
Il y a bien des jeunes peintres qui creusent la terre, prennent de la terre et ensuite exposent cette terre dans une galerie de peinture.
C’est bête et ça prouve le manque d’imagination technique.
C’est intéressant qu’ils aient l’envie de changer le sujet à un point tel qu’ils en arrivent à ça : arracher un morceau de terre et le mettre sur un socle.
Mais ce qu’il faudrait, c’est que la « force » avec laquelle ils arrachent la terre se « retourne ».
Que le morceau de terre soit arraché, oui, mais qu’il soit arraché à leur système personnel et fait avec leur imagination technique.
\vspace*{\fill}
\end{document}


